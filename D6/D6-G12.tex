\documentclass{article}
\usepackage{hyperref}
\usepackage[utf8]{inputenc}
\usepackage[margin=12.7mm]{geometry}
\usepackage{helvet}
\usepackage{graphicx}
\usepackage{color}
\usepackage[parfill]{parskip}
\usepackage{titlesec}
\graphicspath{{./tabella/}}
\titleformat*{\section}{\LARGE\bfseries}
\titleformat*{\subsection}{\Large\bfseries}
\renewcommand{\familydefault}{\sfdefault}
\author{G12}
\title{Report finale G12}
\date{}
\begin{document}
   \maketitle 
   \tableofcontents
   \clearpage  
   \section{Organizzazione del lavoro}
   Il project leader escogitava un piano di lavoro settimanale e poi si discuteva assieme riguardo alla possibilità di metterlo in atto in base agli impegni di ogni componente del gruppo. Spesso ci si suddivideva per macro argomenti il deliverable e poi se qualcuno finiva prima la sua parte continuava con anche delle parti di un altro componente, dopo averne discusso assieme, in modo da non essere mai in ritardo.\\
   La comunicazione è stata un elemento chiave di questi tre mesi di lavoro ai vari deliverable. Solitamente la domenica il team leader proponeva il piano settimanale, ma poi ci si teneva costantemente in contatto in modo che tutti sapessero a che punto ognuno fosse arrivato con il proprio lavoro. In questo modo abbiamo potuto intervenire tempestivamente quando qualcuno era in difficoltà, lavorando insieme per non trovarci a fare qualcosa velocemente per riuscire a consegnare in tempo.\\
   Siamo partiti tutti da zero in quanto a competenze utili per questo corso, quindi non ci siamo organizzati in alcun modo in base alle conoscenze. Ci siamo invece organizzati in base agli impegni e problemi personali di ognuno per riuscire a consegnare ogni singolo deliverable completo e senza trascurarne alcuna parte, permettendo ad ognuno di dare un suo contributo in base alle proprie possibilità e senza creare ansia o stress.\\
   Ogni volta che c’era una parte di un deliverable che poteva poi condizionare altre parti, ad esempio la stesura di requisiti, use cases, user stories, ecc. ne discutevamo tutti e tre, in presenza o a distanza, per trovare un punto di accordo che non avrebbe in futuro svantaggiato nessuno.
   È anche successo che dovessimo essere in due, o magari anche in tre, a lavorare a una singola parte di un deliverable quando ci risultava troppo difficile da gestire singolarmente. Questo è successo ad esempio con lo use-case diagram, class diagram, OCL, e per la maggior parte del D5.
   \section{Ruoli e attività}
   \begin{tabular}{ |p{3cm}|p{15 cm}| } 
      \hline
      Componente del team & Principali attività \\
      \hline 
      Daniele Stella Team leader & Si è occupato dell’organizzazione generale del lavoro e delle decisioni in quanto a tempistiche e scadenze dei compiti di ognuno, mantenendo come già detto grande flessibilità in caso di problemi personali di un componente. Si è occupato della stesura dei documenti, per spiegare le sezioni e unire le varie parti svolte dai componenti del gruppo. Si è impegnato molto in ogni deliverable, e soprattutto si occupato in modo assai attivo per quanto riguarda i deliverable D4 e D5. Per il D5 si è impegnato molto per imparare JavaScript, che non aveva mai utilizzato precedentemente, per poi sviluppare le varie API, la documentazione e il back end \\
      \hline 
      Alessio Blascovich & Si è occupato della stesura in LaTeX di ogni documento (partendo da un file condiviso Google Docs). Ha contribuito attivamente alla realizzazione dei vari diagrammi utilizzati, soprattutto nel D3 e D5. Si è preoccupato di studiare bene OCL per arricchire il class diagram del D4. È stato inoltre di grandissimo aiuto a Daniele durante la stesura di API e backend per il confronto fornito ed essersi assicurato che tutto funzionasse correttamente, svolgendo molte prove di testing. Ha inoltre aiutato molto nella realizzazione del front end \\ 
      \hline
      David Stanicel & Ha fornito grande aiuto nei momenti in cui bisognava prendere delle decisioni e per controllare la correttezza di quanto scritto nei documenti in LaTeX prima della consegna. Si è dimostrato molto disponibile a sostituire un altro componente quando c’erano stati dei contrattempi durante la stesura del D4. Ha contribuito in particolare nei deliverable D2. Per il D5 si è occupato, per la maggior parte in autonomia, del front end.\\
      \hline
   \end{tabular}
   \section{Carico e distribuzione del lavoro}
   Dall'analisi del log risulta il seguente carico di lavoro:\\
   \includegraphics[scale=0.5]{tabella ore log.png}\\
   Si può notare che ci sono parecchi squilibri. Ci teniamo a precisare che non sono propriamente veritiere quelle ore di lavoro, David e Alessio spesso si sono dimenticati di compilare il log.\\
   Ad esempio nel D2 David ha lavorato almeno 3/4 ore anche se nel log ha 0; Alessio invece nel D5 ha lavorato qualche ora in più dato che i primi giorni si dimenticò di compilare il log.\\
   La giustificazione di questi squilibri, oltre alle dimenticanze di compilazione, è che Daniele avendo l’obiettivo di raggiungere il massimo voto possibile in ogni corso che frequenta, era sempre disposto a mettere da parte altri impegni per lavorare il massimo possibile ai deliverable fino a quando riteneva fossero sufficientemente dettagliati e completi. Questo non vuol dire che non ha fatto lavorare gli altri, o che gli altri non avessero voglia di fare, semplicemente si è sempre proposto di fare i compiti che richiedevano più tempo o magari più impegno di preparazione (come ad esempio JS nel D5, o gran parte del class diagram nel D4). Comunque ogni decisione in quanto a suddivisione del lavoro all’interno del gruppo è sempre stata presa tutti assieme e ognuno aveva voce in capitolo. Perciò nessuno è stato costretto a lavorare di più o di meno di altri componenti del gruppo.
   \section{Criticità}
   Essendo studenti al primo semestre del secondo anno di informatica senza alcuna conoscenza pregressa di informatica da un istituto superiore, tutto ciò che abbiamo visto durante il corso è stata una novità. Questo ci ha fatto riscontrare alcuni problemi soprattutto con i deliverable D4 e D5 per cui era richiesta più conoscenza “informatica”. Probabilmente questo è accaduto anche perché il progetto che dovevamo realizzare a volte richiedeva di sapere cose che non avevamo trattato durante il corso, come certi elementi nel class diagram, o molti aspetti e funzionalità che non eravamo in grado di implementare in JavaScript. Nonostante questo siamo riusciti con una buona organizzazione, impegno, e il prezioso aiuto degli esercitatori, a realizzare dei deliverable che crediamo essere completi e non sentono alcuna mancanza di conoscenze. 
   \section{Autovalutazione}
   Crediamo di aver lavorato bene, nonostante alcune difficoltà. Siamo consapevoli che alcune parti non sono state svolte alla perfezione. Abbiamo deciso di utilizzare LaTeX nonostante nessuno di noi lo sapesse per poter produrre dei documenti che potevano sembrare un po’ più professionali, sicuramente facendo qualche errore, ma crediamo si possa vedere un netto miglioramento confrontando i primi deliverable con gli ultimi, quindi pensiamo di aver imparato qualcosa anche sotto quel punto di vista. Il Front-end nel documento D5 è stato abbastanza trascurato, non avevamo mai realizzato nulla in HTML e CSS quindi Alessio e David hanno provato ad apprendere il massimo possibile nel poco tempo disponibile per poter consegnare qualcosa di funzionante e che permettesse di provare le varie API realizzate.
   Sulla base di queste considerazioni, e quanto detto sopra, la nostra autovalutazione è:


   
   \begin{tabular}{|c|c|}
      \hline
      Nome & Voto\\
      \hline
      Daniele Stella & 30 \\
      \hline
      Alessio Blascovich & 26\\
      \hline
      David Stanicel & 23\\
      \hline
   \end{tabular}
\end{document}